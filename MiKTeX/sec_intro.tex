\sectioncentered*{ПЕРЕЧЕНЬ УСЛОВНЫХ ОБОЗНАЧЕНИЙ, СИМВОЛОВ И ТЕРМИНОВ}
\addcontentsline{toc}{section}{Перечень условных обозначений, символов и терминов}
\setcounter{page}{6}
\setlength{\parindent}{0ex}
В настоящей пояснительной записке применяются следующие определения и сокращения.

ПО~--- программное обеспечение.

ОС~--- операционная система.

ZIP~--- формат архивации файлов.

UX~--- дизайн взаимодействия с пользователем.

UI~--- пользовательский интерфейс.

API~--- Application Programming Interfaces.

LLVM~--- Low Level Virtual Machine.

NDK~--- Native Development Kit.

JVM~--- Java Virtual Machine.

XML~--- eXtensible Markup Language.

СУБД~--- база данных.

SQL~--- structured query language~--- язык структурированных запросов.

ACID~--- требования к транзакционной системе, обеспечивающие наиболее надёжную и предсказуемую её работу.

NoSQL~--- not only SQL~--- не только SQL.

JSON~--- текстовый формат обмена данными,основанный на JavaScript.

HTTP~--- HyperText Transfer Protocol.

ПС~--- программное средство.

MVVM~--- Model View ViewModel.

ВИ~--- вариант использования.
\setlength{\parindent}{\fivecharsapprox}
\newpage


\sectioncentered*{Введение}
\addcontentsline{toc}{section}{Введение}
\label{sec:intro}

Темой дипломного проекта является «Android-приложение: Агрегатор игровых магазинов».

Развитие информационных технологий и игрового рынка изменяет процесс взаимодействия между пользователями и дистрибьюторами. С каждым годом появляется всё больше игровых магазинов, а их предложения начинают пересекаться, либо сильно разниться. 

Если обратиться к рейтингу посещаемости игровых магазинов по версии Alexa Traffic Rank \cite{web0} за 2021 год, можно заметить что пользователи интересуются всё большим количеством магазинов, как пример Epic Games Store ворвался на вторую строчку по траффику среди игровых магазинов всего за 2 года, что добавляет пользователям ещё больше выбора.
Востребованность игровой отрасли можно также объяснить эпидемией коронавируса, множество локдаунов и карантинов привели к изоляции большой части общества у себя дома, что вызывает большой рост игровой индустрии в целом и магазинов как её медиаторов.
Повышение количества магазинов также связывается с развитием индустрии в целом, еще в 90-ых годах было невозможно представить, что какая-то конкретная игровая студия откроет свой собственный сервис по продаже игр со своими скидами, бонусами и эксклюзивными предложениями.

Из-за глобальной пандемии COVID-19 и в следствии изоляции и отсутствия у большинства людей капитала как было раньше, образовался спрос на скидки, разные выгодные предложения, которые позволяют потреблять контент за сравнительно малые деньги, в следствии этого и возникла идея создать удобное мобильное приложение для отслеживания скидок и предложений по играм с разных игровых магазинов, а так же поиском этих самых предложений и игр.
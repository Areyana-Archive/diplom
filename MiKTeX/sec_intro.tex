\sectioncentered*{ПЕРЕЧЕНЬ УСЛОВНЫХ ОБОЗНАЧЕНИЙ, СИМВОЛОВ И ТЕРМИНОВ}
\addcontentsline{toc}{section}{Перечень условных обозначений, символов и терминов}
\setcounter{page}{6}
В настоящей пояснительной записке применяются следующие определения и сокращения.

API~--- Application Programming Interfaces.

JSON~---текстовый формат обмена данными,основанный на JavaScript.

БД~--- база данных.

ПО~--- программное обеспечение.

MVC~--- Model View Controller.

MVVM~--- Model View ViewModel

NoSQL~--- not only SQL~--- не только SQL.

SQL~--- structured query language~--- язык структурированных запросов.

REST~--- Representation State Transfer~--- передача состояния представления.
\newpage


\sectioncentered*{Введение}
\addcontentsline{toc}{section}{Введение}
\label{sec:intro}

Развитие информационных технологий и игрового рынка изменяет процесс взаимодействия между пользователями и дистрибьюторами. С каждым годом появляется всё больше игровых магазинов, а их предложения начинают пересекаться, либо сильно разниться. 

Если обратиться к рейтингу посещаемости игровых магазинов по версии Alexa Traffic Rank за 2021 год, можно заметить что пользователи интересуются всё большим количеством магазинов, как пример Epic Games Store ворвался на вторую строчку по траффику среди игровых магазинов всего за 2 года, что добавляет пользователям ещё больше выбора.
Востребованность игровой отрасли можно также объяснить эпидемией коронавируса, множество локдаунов и карантинов привели к изоляции большой части общества у себя дома, что вызывает большой рост игровой индустрии в целом и магазинов как её медиаторов.
Повышение количества магазинов также связывается с развитием индустрии в целом, еще в 90-ых годах было невозможно представить, что какая-то конкретная игровая студия откроет свой собственный сервис по продаже игр со своими скидами, бонусами и эксклюзивными предложениями.

Из-за глобальной пандемии COVID-19 и в следствии изоляции и отсутствия у большинства людей капитала как было раньше, образовался спрос на скидки, разные выгодные предложения, которые позволяют потреблять контент за сравнительно малые деньги, в следствии этого и возникла идея создать удобное мобильное приложение для отслеживания скидок и предложений по играм с разных игровых магазинов, а так же поиском этих самых предложений и игр.
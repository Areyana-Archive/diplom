\sectioncentered*{Реферат}
\thispagestyle{empty}
%%
%% ВНИМАНИЕ: этот реферат не соответствует СТП-01 2013
%% пример оформления реферата смотрите здесь: http://www.bsuir.by/m/12_100229_1_91132.docx 
%%

ANDROID-ПРИЛОЖЕНИЕ: АГРЕГАТОР ИГРОВЫХ МАГАЗИНОВ : дипломный проект / И. В. Демидович. - Минск : БГУИР, 2021, -~\pageref*{LastPage} с., чертежей (плакатов) - 6 л. формата А4.

Пояснительная записка ~\pageref*{LastPage} с., \totfig{}~рисунков, \tottab{}~таблиц, \toteq{}~формул и \totref{} источников.

Целью дипломного проекта является разработка удобного в использовании приложения для поиска скидок на видеоигры.
Основной особенностью является наличие большого количества магазинов и фильтрации по ним.

Для достижения цели дипломного проекта были рассмотрены различные варианты для реализации поставленной цели, рассмотрены аналогичные приложения, их плюсы и минусы, выработаны функциональные и нефункциональные требования.

Было разработано мобильное приложение. 
Данное приложение позволяет пользователю находить лучшие предложения среди множества магазинов в мире.
Разработанное приложение позволяет добавлять в закладки понравившиеся игры и предложения по ним, так же имеется синхронизация с удалёнными серверами.
Также реализованы система нотификаций, ночной режим и система регистрации.

В разделе технико-экономического обоснования был произведён расчёт затрат на создание ПО и прибыли от разработки, получаемой разработчиком. Проведённые расчёты показали экономическую целесообразность проекта.
\clearpage

\section{Резюме проекта}
\label{sec:ttt}
 
Дипломный проект создаётся с целью предоставление конечным пользователям возможности находить скидки на игры среди всех магазинов в быстрой и простой форме через удобное мобильное приложение, агрегируя множество игровых магазинов. Приложение позволит пользователям экономить на покупке игр, путем поиска лучших предложений, установки на них напоминаний и тд.

Приложение разрабатывается как мобильное приложение на платформе Android с приминением последних библиотек из Android SDK, языка программирования Kotlin, субд SQLlite в качестве локальной и NoSQL в качестве удаленной бд. Так же будут подключены сторонние сервисы в виде Firebase (возможно будут подключены и другие).

\subsubsection{Нефункциональные требования}
\begin{itemize}
 \item мобильное приложение должно работать на большом количестве мобильных устройств, минимальная версия Android SDK 23+;
 \item поддержка новейшего Android API;
 \item сопровождение более старых версий Android.
\end{itemize}
 
\subsubsection{Функциональные требования}
\begin{itemize}
  \item поддержка множества магазинов;
  \item возможность фильтровать поиск по множеству фильтров;
  \item вкладка избранное, возможность добавлять игры или предложения в избранное;
  \item вкладка исследование, главная вкладка с предложениями которые ранжирует само приложение;
  \item просмотр дополнительной информации (рейтинг, самая высокая стоимость, самая низкая стоимость и тд);
  \item возможность перехода на сторонние магазины для покупки товара;
  \item социальные функции (шейринг, отображение графики, ведение статистики пользователя).
\end{itemize}
 
\subsection{Перспективы развития приложения}
В дальнейшем приложение может быть улучшено при помощи введения дополнительных возможностей:
\begin{itemize}
  \item подключение дополнительных функций для разных магазинов;
  \item собственная библиотека игр;
  \item инетграция прямых трансляций по определенной игре;
  \item расширение списка магазинов;
  \item добавление большего числа социальных функций;
  \item интеграция точечных сервисов (например Steam API);
  \item календарь релиза игр.
\end{itemize}

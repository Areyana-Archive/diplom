% Зачем: Изменение надписи для списка литературы
% Почему: Пункт 2.8.1 Требований по оформлению пояснительной записки.
\renewcommand{\bibsection}{\sectioncentered*{Cписок использованных источников}}
\phantomsection\pagebreak% исправляет нумерацию в документе и исправляет гиперссылки в pdf
\addcontentsline{toc}{section}{Cписок использованных источников}

% Зачем: Печать списка литературы. База данных литературы - файл bibliography_database.bib
\bibliography{bibliography_database}

\newpage
\phantomsection
\addcontentsline{toc}{section}{Приложение А (обязательное) Листинг программного обеспечения}
\begin{center}
    \textbf{\MakeUppercase{Приложение А}\\
    (обязательное)\\
    Листинг программного обеспечения }
\end{center}

 \newcounter{alphasect}

 \renewcommand\thesection{%
 \ifnum\value{alphasect}=1%
A%%
 \else
\ifnum\value{alphasect}=2%
B%%
\else
\ifnum\value{alphasect}=3%
C%%
\else
\ifnum\value{alphasect}=4%
D%%
\else
 \arabic{section}%%
 \fi\fi\fi\fi}%

 \newenvironment{asection}{%
 \setcounter{alphasect}{1}%%
 }{%
 \setcounter{alphasect}{0}%%
 }%

 \newenvironment{bsection}{%
 \setcounter{alphasect}{2}%%
 }{%
 \setcounter{alphasect}{0}%%
 }%

\begin{asection}

\setlength{\parindent}{0ex}
\begin{lstlisting}[language=Java,label={lst:add:a_1}, caption={BaseActivity}]
abstract class BaseActivity<T : BaseViewModel> : BaseActivity() {
    protected abstract val viewModel: T

    private lateinit var progressDialog: ProgressDialog

    private lateinit var errorDialog: MaterialDialog

    override fun onCreate(savedInstanceState: Bundle?) {
        super.onCreate(savedInstanceState)
        setUpDialogs()
        observeUiChanges()
    }

    private fun setUpDialogs() {
        progressDialog = ProgressDialog()
        errorDialog = MaterialDialog(this).apply {
            positiveButton(text = "Ok")
        }
    }

    private fun observeUiChanges() {
        viewModel.apply {
            progress.observe(
                this@BaseActivity,
                { progress ->
                    handleProgress(progress = progress)
                }
            )

            errorEvent.observe(
                this@BaseActivity,
                { error ->
                    handleAhriError(ahriError = error)
                }
            )
        }
    }

    @Synchronized
    fun handleProgress(
        fragmentManager: FragmentManager = supportFragmentManager,
        progress: Boolean
    ) {
        if (progress) {
            if (!progressDialog.isAdded) {
                progressDialog.show(fragmentManager, ProgressDialog::class.java.canonicalName)
                fragmentManager.executePendingTransactions()
            }
        } else {
            if (progressDialog.isAdded) {
                progressDialog.dismissAllowingStateLoss()
                fragmentManager.executePendingTransactions()
            }
        }
    }

    @Synchronized
    fun handleAhriError(
        fragmentManager: FragmentManager = supportFragmentManager,
        ahriError: AhriError
    ) {
        handleError(fragmentManager, ahriError, ahriError.retryAction)
    }

    private fun handleError(
        fragmentManager: FragmentManager = supportFragmentManager,
        error: AhriError,
        retryAction: (() -> Unit)?
    ) {
        errorDialog.show {
            message(text = error.exception.message)
        }
    }
}
\end{lstlisting}
%\hfill \break
\begin{lstlisting}[language=Java,label={lst:add:a_2}, caption={BaseFragment}]
abstract class BaseFragment<T : BaseViewModel> : BaseFragment() {
    protected abstract val viewModel: T

    override fun onActivityCreated(savedInstanceState: Bundle?) {
        super.onActivityCreated(savedInstanceState)
        observeUiEvents()
    }

    override fun onDestroyView() {
        hideKeyboard()
        requireActivity().currentFocus?.hideKeyboard()
        super.onDestroyView()
    }

    private fun observeUiEvents() {
        viewModel.apply {
            progress.observe(
                viewLifecycleOwner,
                { progress ->
                    (activity as BaseActivity<*>).handleProgress(
                        fragmentManager = childFragmentManager,
                        progress = progress
                    )
                }
            )

            errorEvent.observe(
                viewLifecycleOwner,
                { error ->
                    (activity as BaseActivity<*>).handleAhriError(
                        fragmentManager = childFragmentManager,
                        ahriError = error
                    )
                }
            )
        }
    }

    fun checkPermission(permission: String, requestPermissionCode: Int): Boolean {
        return if (ContextCompat.checkSelfPermission(requireContext(), permission) != PackageManager.PERMISSION_GRANTED) {
            if (shouldShowRequestPermissionRationale(permission)) {
                requestPermissions(arrayOf(permission), requestPermissionCode)
            } else {
                requestPermissions(arrayOf(permission), requestPermissionCode)
            }

            false
        } else {
            true
        }
    }

    protected fun openGoToSettingPermissionDialog(activityResultLauncher: ActivityResultLauncher<Intent>) {
        MaterialDialog(requireContext()).apply {
            title(res = R.string.base_dialog_title_runtime_permission_is_denied)
            message(text = getString(R.string.base_dialog_text_go_to_application_settings))
            cancelable(false)
            positiveButton(
                res = R.string.action_go_to_settings,
                click = {
                    val intent = Intent(ACTION_APPLICATION_DETAILS_SETTINGS)
                    val uri = Uri.fromParts("package", requireContext().packageName, null)
                    intent.data = uri
                    activityResultLauncher.launch(intent)
                }
            )
            negativeButton(res = R.string.action_cancel, click = { /* Nothing */ })
        }.show()
    }
}
\end{lstlisting}
%\hfill \break
\begin{lstlisting}[language=Java,label={lst:add:a_3}, caption={BaseViewModel}]
open class BaseViewModel : ViewModel(), CoroutineScope {

    val progress: MutableLiveData<Boolean> = MutableLiveData()

    override val coroutineContext = SupervisorJob() + Dispatchers.Main

    val errorEvent: LiveEvent<AhriError> = LiveEvent()

    suspend fun handleLoading(
        scope: CoroutineScope = viewModelScope,
        progress: MutableLiveData<Boolean> = this@BaseViewModel.progress,
        actionUseCase: Any? = null,
        block: suspend () -> Unit,
    ) {
        handleLoading(
            progress = progress,
            block = block,
            retryAction = {
                scope.launch {
                    handleLoading(scope, progress, actionUseCase, block)
                }
            }
        )
    }

    private suspend fun handleLoading(
        progress: MutableLiveData<Boolean> = this@BaseViewModel.progress,
        block: suspend () -> Unit,
        retryAction: () -> Unit
    ) {
        handleErrors(
            block = {
                try {
                    progress.value = true
                    block()
                } finally {
                    progress.value = false
                }
            },
            retryAction = retryAction
        )
    }

    suspend fun handleErrors(
        scope: CoroutineScope = viewModelScope,
        block: suspend () -> Unit,
    ) {
        handleErrors(
            block = block,
            retryAction = {
                scope.launch {
                    handleErrors(scope, block)
                }
            }
        )
    }

    private suspend fun handleErrors(
        block: suspend () -> Unit,
        retryAction: () -> Unit
    ) {
        try {
            block()
        } catch (ex: Throwable) {
            errorEvent.value = AhriError(ex, retryAction)

            Timber.tag("debug").e(ex.message ?: "")
        }
    }
}
\end{lstlisting}
%\hfill \break
\begin{lstlisting}[language=Java,label={lst:add:a_4}, caption={Koin}]
var gatewayModule = module {
    single<PreferencesGateway> {
        PreferencesGatewayImpl(preferenceDataSource = get())
    }
    single<StoresGateway> {
        StoresGatewayImpl(
            storesDataSource = get(),
            networkService = get(),
            storesDao = get()
        )
    }
    single<DealsGateway> {
        DealsGatewayImpl(
            dealsDataSource = get(),
            storesGateway = get(),
            favoriteDao = get(),
            dealDao = get()
        )
    }
    single<GamesGateway> {
        GamesGatewayImpl(
            gamesDataSource = get(),
            networkService = get(),
            storesGateway = get(),
            favoriteGameDao = get(),
            gameDao = get()
        )
    }
    single<AuthGateway> {
        AuthGatewayImpl()
    }
    single<StoriesGateway> {
        StoriesGatewayImpl()
    }
    single<BannersGateway> {
        BannersGatewayImpl()
    }
    single<RssGateway> {
        RssGatewayImpl(rssApi = get())
    }
}
\end{lstlisting}
%\hfill \break
\begin{lstlisting}[language=Java,label={lst:add:a_5}, caption={AndroidManifest.xml}]]
<?xml version="1.0" encoding="utf-8"?>
<manifest xmlns:android="http://schemas.android.com/apk/res/android"
        package="com.lust.ahri">

    <uses-permission android:name="android.permission.ACCESS_NETWORK_STATE" />
    <uses-permission android:name="android.permission.READ_PHONE_STATE" />
    <uses-permission android:name="android.permission.INTERNET" />
    <uses-permission android:name="android.permission.CAMERA"/>
    <uses-permission android:name="android.permission.READ_EXTERNAL_STORAGE" />
    <uses-permission android:name="android.permission.WRITE_EXTERNAL_STORAGE" />

    <application
            android:name=".App"
            android:allowBackup="true"
            android:icon="@mipmap/ic_launcher"
            android:label="@string/app_name"
            android:roundIcon="@mipmap/ic_launcher_round"
            android:supportsRtl="true"
            android:theme="@style/AppTheme">

        <activity
                android:name="com.lust.ahri.view.main.MainActivity"
                android:label="@string/app_name"
                android:screenOrientation="portrait" />

        <activity
                android:name=".view.auth.AuthActivity"
                android:screenOrientation="portrait" />

        <activity
                android:name=".view.onboarding.OnboardingActivity"
                android:screenOrientation="portrait">
            <intent-filter>
                <action android:name="android.intent.action.MAIN" />
                <category android:name="android.intent.category.LAUNCHER" />
            </intent-filter>
        </activity>

        <service
                android:name=".core.service.firebase.AhriMessagingService">
            <intent-filter>
                <action android:name="com.google.firebase.MESSAGING_EVENT"/>
            </intent-filter>
        </service>

        <provider
                android:name="br.com.mauker.materialsearchview.db.HistoryProvider"
                android:authorities="br.com.mauker.materialsearchview.searchhistorydatabase"
                android:exported="false"
                android:protectionLevel="signature"
                android:syncable="true"/>

        <provider
                android:name="androidx.core.content.FileProvider"
                android:authorities="${applicationId}.fileprovider"
                android:exported="false"
                android:grantUriPermissions="true">
            <meta-data
                    android:name="android.support.FILE_PROVIDER_PATHS"
                    android:resource="@xml/file_paths" />
        </provider>

    </application>

</manifest>
\end{lstlisting}
%\hfill \break
\begin{lstlisting}[language=Java,label={lst:add:a_6}, caption={AuthSignInViewModel}]
class AuthSignInViewModel(
    private val validateEmail: ValidateEmail,
    private val validatePassword: ValidatePassword,
    private val signIn: SignIn
) : BaseViewModel() {
    val password: MutableLiveData<String> = MutableLiveData()
    val email: MutableLiveData<String> = MutableLiveData()

    private val _initSignInEvent: LiveEvent<Nothing> = LiveEvent()
    val initSignInEvent: LiveData<Nothing>
        get() = _initSignInEvent

    private val _openResetPasswordEvent: LiveEvent<Nothing> = LiveEvent()
    val openResetPasswordEvent: LiveData<Nothing>
        get() = _openResetPasswordEvent

    val isSignUpDataValid: LiveData<Boolean> = combineLatest(email, password) {
        val email = email.value
        val password = password.value

        if (email != null && password != null) {
            return@combineLatest email.isNotEmpty() && password.isNotEmpty()
        }

        return@combineLatest false
    }

    private val _passwordValidationEvent: LiveEvent<ValidatePassword.Result> =
        LiveEvent()
    val passwordValidationEvent: LiveData<ValidatePassword.Result>
        get() = _passwordValidationEvent

    private val _emailValidationEvent: LiveEvent<ValidateEmail.Result> =
        LiveEvent()
    val emailValidationEvent: LiveData<ValidateEmail.Result>
        get() = _emailValidationEvent

    private fun validatePassword(): Boolean {
        val result = validatePassword(password.value)
        _passwordValidationEvent.emit(result)
        return when (result) {
            ValidatePassword.Result.Valid -> true
            ValidatePassword.Result.NotValid, ValidatePassword.Result.IsEmpty -> false
        }
    }

    private fun validateEmail(): Boolean {
        val result = validateEmail(email.value)
        _emailValidationEvent.emit(result)
        return when (result) {
            ValidateEmail.Result.Valid -> true
            ValidateEmail.Result.IsEmpty, ValidateEmail.Result.NotValid -> false
        }
    }

    private fun validateSignInInputs(): Boolean {
        return validatePassword() && validateEmail()
    }

    fun forgotPassword() {
        _openResetPasswordEvent.emit()
    }

    fun logIn() {
        validateSignInInputs().applyIfTrue {
            viewModelScope.launch {
                handleLoading {
                    signIn(email.value!!, password.value!!)
                    _initSignInEvent.emit()
                }
            }
        }
    }
}
\end{lstlisting}
%\hfill \break
\begin{lstlisting}[language=Java,label={lst:add:a_10}, caption={DealsPagingSource}]
class DealsPagingSource(
    private val getDeals: GetDeals,
    private val title: String? = null,
    private val ids: List<String>? = null,
    private val metacritic: Int? = null
) : PagingSource<Int, Deal>() {
    override fun getRefreshKey(state: PagingState<Int, Deal>): Int? {
        return state.anchorPosition
    }

    override suspend fun load(params: LoadParams<Int>): LoadResult<Int, Deal> {
        return try {
            val startNumber = params.key ?: 0
            val endNumber = startNumber + 1
            val response = getDeals(startNumber, title, ids, metacritic)

            LoadResult.Page(
                data = response,
                prevKey = null,
                nextKey = if (response.isEmpty()) null else endNumber
            )
        } catch (e: Exception) {
            LoadResult.Error(e)
        }
    }
}
\end{lstlisting}
%\hfill \break
\begin{lstlisting}[language=Java,label={lst:add:a_7}, caption={GestureDetectorCompat}]
class AuthSignInViewModel(
    private val validateEmail: ValidateEmail,
    private val validatePassword: ValidatePassword,
    private val signIn: SignIn
) : BaseViewModel() {
    val password: MutableLiveData<String> = MutableLiveData()
    val email: MutableLiveData<String> = MutableLiveData()

    private val _initSignInEvent: LiveEvent<Nothing> = LiveEvent()
    val initSignInEvent: LiveData<Nothing>
        get() = _initSignInEvent

    private val _openResetPasswordEvent: LiveEvent<Nothing> = LiveEvent()
    val openResetPasswordEvent: LiveData<Nothing>
        get() = _openResetPasswordEvent

    val isSignUpDataValid: LiveData<Boolean> = combineLatest(email, password) {
        val email = email.value
        val password = password.value

        if (email != null && password != null) {
            return@combineLatest email.isNotEmpty() && password.isNotEmpty()
        }

        return@combineLatest false
    }

    private val _passwordValidationEvent: LiveEvent<ValidatePassword.Result> =
        LiveEvent()
    val passwordValidationEvent: LiveData<ValidatePassword.Result>
        get() = _passwordValidationEvent

    private val _emailValidationEvent: LiveEvent<ValidateEmail.Result> =
        LiveEvent()
    val emailValidationEvent: LiveData<ValidateEmail.Result>
        get() = _emailValidationEvent

    private fun validatePassword(): Boolean {
        val result = validatePassword(password.value)
        _passwordValidationEvent.emit(result)
        return when (result) {
            ValidatePassword.Result.Valid -> true
            ValidatePassword.Result.NotValid, ValidatePassword.Result.IsEmpty -> false
        }
    }

    private fun validateEmail(): Boolean {
        val result = validateEmail(email.value)
        _emailValidationEvent.emit(result)
        return when (result) {
            ValidateEmail.Result.Valid -> true
            ValidateEmail.Result.IsEmpty, ValidateEmail.Result.NotValid -> false
        }
    }

    private fun validateSignInInputs(): Boolean {
        return validatePassword() && validateEmail()
    }

    fun forgotPassword() {
        _openResetPasswordEvent.emit()
    }

    fun logIn() {
        validateSignInInputs().applyIfTrue {
            viewModelScope.launch {
                handleLoading {
                    signIn(email.value!!, password.value!!)
                    _initSignInEvent.emit()
                }
            }
        }
    }
}
\end{lstlisting}
%\hfill \break
\begin{lstlisting}[language=Java,label={lst:add:a_8}, caption={UriBindingAdapter}]
object UriBindingAdapter {
    @JvmStatic
    @BindingAdapter(value = ["image_uri"])
    fun setUri(imageView: ImageView, url: Uri?) {
        Glide.with(imageView)
            .load(url)
            .into(imageView)
    }

    @JvmStatic
    @BindingAdapter(value = ["image_uri", "image_placeholder"])
    fun setUriWithPlaceholder(imageView: ImageView, url: Uri?, placeholder: Drawable) {
        Glide.with(imageView)
            .load(url)
            .placeholder(placeholder)
            .fallback(placeholder)
            .into(imageView)
    }

    @JvmStatic
    @BindingAdapter(value = ["image_uri", "image_placeholder", "image_radius"])
    fun setUriWithPlaceholderRadius(
        imageView: ImageView,
        url: Uri?,
        placeholder: Drawable,
        imageRadius: Int
    ) {
        Glide.with(imageView)
            .load(url)
            .transform(CenterCrop(), RoundedCorners(imageRadius.dp))
            .placeholder(placeholder)
            .fallback(placeholder)
            .into(imageView)
    }

    @JvmStatic
    @BindingAdapter(value = ["image_res", "image_placeholder", "image_radius"])
    fun setResWithPlaceholderRadius(
        imageView: ImageView,
        @DrawableRes res: Int?,
        placeholder: Drawable,
        imageRadius: Int
    ) {
        Glide.with(imageView)
            .load(res)
            .transform(CenterCrop(), RoundedCorners(imageRadius.dp))
            .placeholder(placeholder)
            .fallback(placeholder)
            .into(imageView)
    }

    @JvmStatic
    @BindingAdapter(value = ["circle_image_uri"])
    fun setCircleUri(imageView: ImageView, url: Uri?) {
        url?.let {
            Glide.with(imageView)
                .load(url)
                .circleCrop()
                .into(imageView)
        }
    }

    @JvmStatic
    @BindingAdapter(value = ["circle_image_uri", "placeholder"])
    fun setCircleUriWithPlaceholder(imageView: ImageView, url: Uri?, placeholder: Drawable) {
        url?.let {
            Glide.with(imageView)
                .load(url)
                .circleCrop()
                .placeholder(placeholder)
                .fallback(placeholder)
                .into(imageView)
        }
    }
}
\end{lstlisting}
%\hfill \break
\begin{lstlisting}[language=Java,label={lst:add:a_11}, caption={Методы для взаимодействия с камерой}]
private fun dispatchTakePictureFromCameraIntent() {
        avatarUri =
            getOutputMediaFile("avatar_${OffsetDateTime.now().toInstant().toEpochMilli()}.jpg")
        openTakePictureFromCamera.launch(avatarUri)
    }

    private val openTakePictureFromCamera =
        registerForActivityResult(ActivityResultContracts.TakePicture()) { result ->
            if (result) {
                viewModel.addAvatarEvent(avatarUri!!)
            }
        }

    private val openTakePictureFromGallery =
        registerForActivityResult(ActivityResultContracts.GetContent()) { result ->
            avatarUri = result
            viewModel.addAvatarEvent(avatarUri!!)
        }

    private val checkCameraPermission =
        registerForActivityResult(ActivityResultContracts.RequestPermission()) { result ->
            if (result) {
                dispatchTakePictureFromCameraIntent()
            } else {
                if (!shouldShowRequestPermissionRationale(Manifest.permission.CAMERA)) {
                    openGoToSettingPermissionDialog(openSettingsPermissionAppActivity)
                }
            }
        }

    private val openSettingsPermissionAppActivity: ActivityResultLauncher<Intent> =
        registerForActivityResult(ActivityResultContracts.StartActivityForResult()) {
            checkCameraPermission.launch(Manifest.permission.CAMERA)
        }
\end{lstlisting}
%\hfill \break
\begin{lstlisting}[language=Java,label={lst:add:a_9}, caption={AhriMessagingService}, caption={PreferenceDataSource}]
object UriBindingAdapter {
    @JvmStatic
    @BindingAdapter(value = ["image_uri"])
    fun setUri(imageView: ImageView, url: Uri?) {
        Glide.with(imageView)
            .load(url)
            .into(imageView)
    }

    @JvmStatic
    @BindingAdapter(value = ["image_uri", "image_placeholder"])
    fun setUriWithPlaceholder(imageView: ImageView, url: Uri?, placeholder: Drawable) {
        Glide.with(imageView)
            .load(url)
            .placeholder(placeholder)
            .fallback(placeholder)
            .into(imageView)
    }

    @JvmStatic
    @BindingAdapter(value = ["image_uri", "image_placeholder", "image_radius"])
    fun setUriWithPlaceholderRadius(
        imageView: ImageView,
        url: Uri?,
        placeholder: Drawable,
        imageRadius: Int
    ) {
        Glide.with(imageView)
            .load(url)
            .transform(CenterCrop(), RoundedCorners(imageRadius.dp))
            .placeholder(placeholder)
            .fallback(placeholder)
            .into(imageView)
    }

    @JvmStatic
    @BindingAdapter(value = ["image_res", "image_placeholder", "image_radius"])
    fun setResWithPlaceholderRadius(
        imageView: ImageView,
        @DrawableRes res: Int?,
        placeholder: Drawable,
        imageRadius: Int
    ) {
        Glide.with(imageView)
            .load(res)
            .transform(CenterCrop(), RoundedCorners(imageRadius.dp))
            .placeholder(placeholder)
            .fallback(placeholder)
            .into(imageView)
    }

    @JvmStatic
    @BindingAdapter(value = ["circle_image_uri"])
    fun setCircleUri(imageView: ImageView, url: Uri?) {
        url?.let {
            Glide.with(imageView)
                .load(url)
                .circleCrop()
                .into(imageView)
        }
    }

    @JvmStatic
    @BindingAdapter(value = ["circle_image_uri", "placeholder"])
    fun setCircleUriWithPlaceholder(imageView: ImageView, url: Uri?, placeholder: Drawable) {
        url?.let {
            Glide.with(imageView)
                .load(url)
                .circleCrop()
                .placeholder(placeholder)
                .fallback(placeholder)
                .into(imageView)
        }
    }
}
\end{lstlisting}
\end{asection}
\newpage
\phantomsection
\addcontentsline{toc}{section}{Приложение Б (обязательное) Работа с Firebase}

\begin{center}
    \textbf{\MakeUppercase{Приложение Б}\\
    (обязательное)\\
    Работа с Firebase }
\end{center}

\begin{lstlisting}[language=Java,label={lst:add:firebase}]
class AuthGatewayImpl : AuthGateway {

    private val auth = FirebaseAuth.getInstance()
    private val storage = FirebaseStorage.getInstance()
    private val db = FirebaseFirestore.getInstance()

    override fun isLoggedIn(): Boolean {
        return auth.currentUser != null
    }

    override suspend fun getUserId(): String {
        val user = auth.currentUser!!
        return user.uid
    }

    override fun getAccountData(): Pair<String, String> {
        val user = auth.currentUser!!
        return (user.displayName ?: "No Name") to (user.email ?: "No Email")
    }

    override suspend fun signUp(email: String, password: String, name: String) {
        return suspendCoroutine { cont ->
            auth.createUserWithEmailAndPassword(email, password)
                .addOnCompleteListener {
                    if (it.isSuccessful) {
                        auth.currentUser?.updateProfile(
                            UserProfileChangeRequest.Builder().setDisplayName(name).build()
                        )
                        cont.resume(Unit)
                    } else {
                        cont.resumeWithException(Exception(if (it.exception != null) it.exception!!.message else "Wrong login or password"))
                    }
                }
        }
    }

    override suspend fun signIn(email: String, password: String) {
        return suspendCoroutine { cont ->
            auth.signInWithEmailAndPassword(email, password)
                .addOnCompleteListener {
                    if (it.isSuccessful) {
                        cont.resume(Unit)
                    } else {
                        cont.resumeWithException(Exception("Wrong login or password"))
                    }
                }
        }
    }

    override suspend fun resetPassword(email: String) {
        return suspendCoroutine { cont ->
            auth.sendPasswordResetEmail(email)
                .addOnCompleteListener {
                    if (it.isSuccessful) {
                        cont.resume(Unit)
                    } else {
                        cont.resumeWithException(Exception("Something went wrong"))
                    }
                }
        }
    }

    override fun signOut() {
        auth.signOut()
    }

    private val avatarFileName = "avatar.jpg"

    override suspend fun postAvatar(uri: Uri) {
        return suspendCoroutine { cont ->
            val uid = auth.currentUser?.uid!!
            val userAvatarRef = storage.reference
                .child("users")
                .child(uid)
                .child(avatarFileName)
            userAvatarRef.putFile(uri)
                .addOnCompleteListener { result ->
                    if (result.isSuccessful) {
                        cont.resume(Unit)
                    } else {
                        cont.resumeWithException(Exception("Something went wrong"))
                    }
                }
        }
    }

    override suspend fun deleteAvatar() {
        return suspendCoroutine { cont ->
            val uid = auth.currentUser?.uid!!
            val userAvatarRef = storage.reference
                .child("users")
                .child(uid)
                .child(avatarFileName)
            userAvatarRef.delete()
                .addOnCompleteListener { result ->
                    if (result.isSuccessful) {
                        cont.resume(Unit)
                    } else {
                        cont.resumeWithException(Exception("Something went wrong"))
                    }
                }
        }
    }

    val usersKey: String = "users"
    val nightModeKey: String = "nightMode"
    val notificationsKey: String = "notifications"
    val storesKey: String = "stores"
    val languageKey: String = "language"

    override suspend fun putNightMode(boolean: Boolean) {
        return suspendCoroutine { cont ->
            val uid = auth.currentUser?.uid!!
            val users = db.collection(usersKey).document(uid)
            val map: Map<String, Boolean> = mapOf(nightModeKey to boolean)
            users.set(map, SetOptions.merge()).addOnCompleteListener {
                cont.resume(Unit)
            }.addOnCanceledListener {
                cont.resumeWithException(Exception("No Connection"))
            }
        }
    }

    override suspend fun getNightMode(): Boolean {
        return suspendCoroutine { cont ->
            val uid = auth.currentUser?.uid!!
            val users = db.collection(usersKey).document(uid)
            users.get().addOnCompleteListener {
                if (it.isSuccessful && it.result != null) {
                    val result = it.result!!
                    if (result.contains(nightModeKey)) {
                        cont.resume(result[nightModeKey] as Boolean)
                    } else {
                        cont.resume(false)
                    }
                } else {
                    cont.resumeWithException(Exception("No Connection"))
                }
            }.addOnCanceledListener {
                cont.resumeWithException(Exception("No Connection"))
            }
        }
    }

    override suspend fun putNotification(boolean: Boolean) {
        return suspendCoroutine { cont ->
            val uid = auth.currentUser?.uid!!
            val users = db.collection(usersKey).document(uid)
            val map: Map<String, Boolean> = mapOf(notificationsKey to boolean)
            users.set(map, SetOptions.merge()).addOnCompleteListener {
                cont.resume(Unit)
            }.addOnCanceledListener {
                cont.resumeWithException(Exception("No Connection"))
            }
        }
    }

    override suspend fun getNotification(): Boolean {
        return suspendCoroutine { cont ->
            val uid = auth.currentUser?.uid!!
            val users = db.collection(usersKey).document(uid)
            users.get().addOnCompleteListener {
                if (it.isSuccessful && it.result != null) {
                    val result = it.result!!
                    if (result.contains(notificationsKey)) {
                        cont.resume(result[notificationsKey] as Boolean)
                    } else {
                        cont.resume(false)
                    }
                } else {
                    cont.resumeWithException(Exception("No Connection"))
                }
            }.addOnCanceledListener {
                cont.resumeWithException(Exception("No Connection"))
            }
        }
    }

    override suspend fun putStores(boolean: Boolean) {
        return suspendCoroutine { cont ->
            val uid = auth.currentUser?.uid!!
            val users = db.collection(usersKey).document(uid)
            val map: Map<String, Boolean> = mapOf(storesKey to boolean)
            users.set(map, SetOptions.merge()).addOnCompleteListener {
                cont.resume(Unit)
            }.addOnCanceledListener {
                cont.resumeWithException(Exception("No Connection"))
            }
        }
    }

    override suspend fun getStores(): Boolean {
        return suspendCoroutine { cont ->
            val uid = auth.currentUser?.uid!!
            val users = db.collection(usersKey).document(uid)
            users.get().addOnCompleteListener {
                if (it.isSuccessful && it.result != null) {
                    val result = it.result!!
                    if (result.contains(storesKey)) {
                        cont.resume(result[storesKey] as Boolean)
                    } else {
                        cont.resume(false)
                    }
                } else {
                    cont.resumeWithException(Exception("No Connection"))
                }
            }.addOnCanceledListener {
                cont.resumeWithException(Exception("No Connection"))
            }
        }
    }

    override suspend fun getLanguage(): String {
        return suspendCoroutine { cont ->
            val uid = auth.currentUser?.uid!!
            val users = db.collection(usersKey).document(uid)
            users.get().addOnCompleteListener {
                if (it.isSuccessful && it.result != null) {
                    val result = it.result!!
                    if (result.contains(languageKey)) {
                        cont.resume(result[languageKey] as String)
                    } else {
                        cont.resume("en")
                    }
                } else {
                    cont.resumeWithException(Exception("No Connection"))
                }
            }.addOnCanceledListener {
                cont.resumeWithException(Exception("No Connection"))
            }
        }
    }

    override suspend fun putLanguage(language: String) {
        return suspendCoroutine { cont ->
            val uid = auth.currentUser?.uid!!
            val users = db.collection(usersKey).document(uid)
            val map: Map<String, String> = mapOf(languageKey to language)
            users.set(map, SetOptions.merge()).addOnCompleteListener {
                cont.resume(Unit)
            }.addOnCanceledListener {
                cont.resumeWithException(Exception("No Connection"))
            }
        }
    }

    override suspend fun getAvatar(): Uri? {
        return suspendCoroutine { cont ->
            val uid = auth.currentUser?.uid!!
            val userAvatarRef = storage.reference
                .child("users")
                .child(uid)
                .child(avatarFileName)
            userAvatarRef.downloadUrl
                .addOnCompleteListener { result ->
                    if (result.isSuccessful) {
                        cont.resume(result.result)
                    } else {
                        cont.resume(null)
                    }
                }
        }
    }
}
\end{lstlisting}

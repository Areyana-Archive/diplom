\sectioncentered*{Заключение}
\addcontentsline{toc}{section}{Заключение}
 
В рамках данного дипломного проекта был рассмотрен вопрос синхронизации воспроизведения видео по сети интернет, рассмотрены сервисы, которые предоставляют подобную функциональность, и различные способы синхронизации. Также было разработано веб-приложение, которое предоставляет пользовательский интерфейс для создания комнат и синхронизации процесса воспроизведения видео. Помимо это были реализованы дополнительные возможности: чат, уведомления и список воспроизведения.
 
При разработке приложения использовались современные технологии, методики разработки и архитектурные решения: код приложения разбит на отдельные модули (компоненты), которые выполняют задачи необходимые только для данного компонента. Был изучен и использован сервис Firebase, в частности Firebase Firestore, который позволяет повысить интерактивность приложения и ускорить разработку, минимизируя затраты. Также были изучены и применены различные средства для взаимодействия и получения видео с таких сторонних видеосервисов, как YouTube и Vimeo.
 
В результате цель дипломного проекты была дастигнута.
Было создано программное обеспечение, выполняющее минимально необходимый набор возможностей.
Однако было оставлено множество мест для улучшения разработанного ПО в дальнейшем. К таким улучшениям можно отнести: поддержку прямых трансляций, улучшение при работе с медленным интернет соединением, расширением возможностей чата (отправка изображений, видео и gif-файлов) и расширение возможностей по администрированию комнаты. В дальнейшем планируется улучшить данное приложение, введя дополнительные возможности и улучшая текущие, а также добавить полноценную поддержку мобильных платформ.

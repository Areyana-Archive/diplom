\sectioncentered*{Выводы}
\addcontentsline{toc}{section}{Выводы}
 
В рамках данного дипломного проекта был рассмотрен вопрос поиска лучших скидок в игровой индустрии, работы с сетью и синхронизации данных на сервере с локальной базой данных, рассмотрены приложения и сервисы, которые предоставляют подобную функциональность. Также было разработано мобильное приложение, которое предоставляет пользовательский интерфейс для поиска игр и предложений по различным магазинам, а так же сохранения их в избранное. Помимо это были реализованы дополнительные возможности: уведомления, переходы на сторонние сервисы, смена языка, темная тема, получение последних новостей из мира игр.
 
При разработке приложения использовались современные технологии, методики разработки и архитектурные решения: код приложения разбит на отдельные модули (компоненты), которые выполняют задачи необходимые только для данного компонента. Был изучен и использован сервис Firebase, в частности Firebase Firestore, Firebase Auth, Firebase Storage, которые позволяют повысить интерактивность приложения и ускорить разработку, минимизируя затраты. Также были изучены и применены различные средства для работы с XML и JSON файлами, применены современные технологии в многопоточном программировании под Android системы. Приложение использует новейшие Android Architecture Components и рекомендуемую для них MVVM архитектуру. Данный подход позволит без каких-либо проблем и дальше расширять приложение различными модулями. Так же используется рекомендуемый Google язык программирования Kotlin, что позволит кодовой базе не устаревать и получать стабильные обновления.
 
В результате цель дипломного проекты была достигнута.
Было создано программное обеспечение, выполняющее минимально необходимый набор возможностей.
Однако было оставлено множество мест для улучшения разработанного ПО в дальнейшем. К таким улучшениям можно отнести: поддержку прямых трансляций различных игр, улучшение при работе с медленным интернет соединением, расширение социальных функций (отправка изображений, видео и gif-файлов), добавление больше сторонних магазинов игр, которые сейчас не присутствуют в приложении. Введение машинного обучения для перенастройки ранжирования магазинов исходя из их популярности у пользователей. В дальнейшем планируется улучшить данное приложение, введя дополнительные возможности и улучшая текущие, а также добавить полноценную поддержку режима без интернета и кеширование результатов взаимодействия пользователя с приложением.

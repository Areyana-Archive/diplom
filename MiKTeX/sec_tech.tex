\section{Используемые технологии}
\label{sec:practice:technology_used}
 
\subsection{Синхронизация данных}
При разработке видеоплеера было необходимо определиться с теми технологиями и методами синхронизации и обмена данными, которые доступны для реализации в веб-приложении. Были рассмотрены следующие варианты: стриминг видео, обмен синхронизирующих сообщений между приложениями, обмен синхронизирующих сообщений через сервер.
 
\subsubsection{Стриминг}~\par
Разберёмся с понятием стриминга. В данном случае речь идёт об HTML5-стриминге, также существует понятие HTML5-видео. Различия заключаются в том, то для HTML5-видео используется готовый видео файл, а для HTML5-стриминга постоянный видеопоток (видео на YouTube — это HTML5-видео, а трансляция на Twitch — HTML5-стриминг).
 
Недостатком данного метода является наличие большого количества кодеков, транспортных и видео протоколов, которые имеют свои нюансы, ограничения и проблемы совместимости. Также данный метод не подходит для данного проекта, так как данная система накладывает ограничения на видеофайлы, которые пользователь сможет использовать, потому что сначала их необходимо загрузить на сервер, откуда они будут передаваться остальным пользователям. Это также ограничивает возможности синхронизации, так как при стриминге для обработки запросов плеера (пауза, перемотка) для множества сессий потребуются большие серверные мощности. Также это накладывает ограничения на скорость исходящего интернет соединения пользователя, необходимое для передачи видеопотока на сервер.
 
\subsubsection{Обмен сообщений между приложениями}~\par 
Второй вариант заключается в том, что каждое приложение является независимым и для синхронизации оно обменивается сообщениями о состоянии видео с другими приложениями. 
Данный метод позволяет снизить нагрузку на сервер, так как основные действия будут происходить на устройстве пользователя.
Остаётся вопрос обмена синхронизирующими сообщениями.
Для этих целей подходит технология Web Real-Time Communications. Web Real-Time Communications (WebRTC) - это технология, которая позволяет веб-приложениям и сайтам осуществлять захват и передачу аудио и/или видео потоков. Также имеется возможность обмениваться данными произвольного типа между браузерами, без необходимости посредника в этом процессе. На данный момент реализация WebRTC в современных браузерах неполная, что выражается в различной степени реализованности как функций, так и кодеков WebRTC. WebRTC предлагает ряд интерфейсов для передачи медиа потоков и произвольных данных, контроля соединения, управление идентификацией и многое другое.
 
В основе WebRTC лежит ряд протоколов, которые обеспечивают возможность P2P соединения:
\begin{itemize}
 \item Interactive Connectivity Establishment (ICE);
 \item Session Traversal Utilities for NAT (STUN);
 \item Session Traversal Utilities for NAT (STUN);
 \item Network Address Translation (NAT);
 \item Traversal Using Relays around NAT (TURN);
 \item Session Description Protocol (SDP).
\end{itemize}
 
Network Address Translation (NAT) - механизм, позволяющий заменить внутренний IP адрес и порт в пакетах, отправляемых с локального устройства, на внешний адрес маршрутизатора для доступа во внешнюю сеть. Существует несколько способов трансляции адреса: статический, при котором адреса соотносятся один к одному (это необходимо когда данное устройство должно быть доступно из внешней сети), динамически, когда внутреннему адресу ставиться один из доступных адресов, и перегруженный, при котором адреса нескольких устройств заменяются на один и тот же адрес, но с различными номерами портов. Существует несколько типов NAT, которые способны оказать трудности в процессе установки соединения между устройствами. Из них нам интересен симметричный NAT.
 
Симметричный NAT - трансляция, при которой каждый запрос с определённого внутреннего адреса и порта на конкретный внешний адрес и порт преобразуется в уникальный внешний адрес и порт. Если устройства с одинаковым адресом и портом отправят запрос на разные адреса и порты, им будут присвоены разные адреса. При данном типе NAT компьютер способен получать пакеты только от тех источников, которым он уже отправлял запросы, иначе говоря, запрос из внешней сети от неизвестного отправителя будет проигнорирован.
 
Interactive Connectivity Establishment (ICE) - это технология, используемая для нахождения кратчайшего пути коммуникации между двумя компьютерами. Данный протокол необходим для тех приложений, в которых обработка сообщений через центральный сервер является неэффективным как в плане скорости передачи данных, так и из-за дополнительных финансовых затрат.
 
Для установки соединения необходимо знать свой IP адрес и ограничения NAT. В этом помогает вспомогательный сервер, который реализует Session Traversal Utilities for NAT (STUN) протокол. Если роутер реализует симметричный NAT, то организация соединения по публичным IP адресам становиться невозможной. В таком случае необходимо использовать дополнительный сервер, реализующий Traversal Using Relays around NAT (TURN). Работа данного сервера заключается в пересылке данных в обход симметричного NAT. Этот подход имеет свои издержки и его принято использовать при отсутствии иных способов установить соединение.

Установки соединения между браузерами, используя WebRTC, также является нежелательным методом для данного проекта, потому что для того, чтобы все пользователи могли использовать видеоплеер, необходимо много дополнительной ресурсов на разработку и поддержание системы установки соединения. Минусом данного способа является его ненадёжность и высокие затраты на реализацию механизма обмена сообщений. Также данный метод является ненадёжным в ситуации потери соединения из-за независимости приложений, что является нежелательным для данного дипломного проекта. 
 
\subsubsection{Обмен сообщений с сервером}~\par 
Для реализации поставленной цели было решено использовать метод, при котором приложения будут обмениваться сообщениями через серве. Это позволит избежать потери данных при сохранении приемлемой скорости. Приложения будут точно занть необходимое состояние видео, так как верная информация об этом будет всегда находиться на сервере. Также данный метод позволяет при рассинхронизации принудительно синхронизировать клиенты, так как имеется единый источник информации. Описание способа обмена сообщений через червер представлено в разделе~\ref{tech:database}.

\subsection{Технологии для сервера}
\label{tech:database}
Для того, чтобы упростить разработку приложения, было решено использовать сервис Firebase от компании Google. Данный сервис предлагает множество модулей, которые очень полезны для веб-разработки, создания мобильных или настольных приложений. Данный сервис избавляет от нужды в собственном сервере, так как Firebase реализует практически все возможности, которые необходимы от стандартного сервера. Главным инструментом в составе Firebase является их база данных. На данный момент Firebase предлагает несколько различных вариантов баз данных: Realtime Database и Firestore. Их главное отличие заключается в форме хранения данных и доступа к ним. 

Realtime Database использует JSON файлы для хранения информации. Это является крайне неэффективным, так как для получения определённых данных сначала нужно получить все данные из файла, а только потом можно произвести поиск. Это увеличивает потребность приложения в ресурсах системы, что недопустимо для данного проекта. 

Firestore~--- облачная NoSQL база данных реального времени. Это означает то, что, при использовании данной базы данных, пользователь может "подписаться" на определённые документы. В таком случае при изменении содержимого документа пользователь сразу получит новые данные. Благодаря данному механизму можно быть уверенным, что конечный пользователь получит самую свежую информацию, что является важным для данного проекта. Firestore поддерживает следующие типы данных:

\begin{itemize}
    \item текстовая строка;
    \item числа;
    \item булевые значения;
    \item массивы;
    \item даты;
    \item словари.
\end{itemize}

Так как Firestore является NoSQL базой данных, то для организации данных в ней используются не таблицы, а документы. Отличие заключается в том, что документ не имеет жёсткой структуры, как таблицы, что позволяет хранить в документах данные произвольных типов и менять их значение и структуру по мере необходимости. Данный способ позволяет более гибко взаимодействовать с данными, но лишает дополнительной надёжности, свойственной SQL базам данных.

Для организации документов используются коллекции. Коллекция~--- это просто набор документов. Они не обязаны быть одинаковыми, но данный случай крайне нежелателен. Каждый документы в коллекции имеет идентификатор, уникальный для данной коллекции. В качестве идентификатора может выступать любая подходящая текстовая строка. Документы также способны содержать внутри себя вложенные коллекции, однако данная возможность не особо полезна. Также имеется возможность создания пользовательских функций и триггеров для взаимодействия с базой данных, что позволяет выполнять определённые операции при изменении, создании или удалении документов и коллекций. 

Firebase также предлагает встроенную систему аутентификации пользователей, которая поддерживает различные источники для идентификации: почта, профиль Google и др.

Также плюсом Firebase является возможность его интеграции с другими сервисами компании Google и дополнительная защита от прекращения работы собственного сервера.

\subsection{Недостатки Firebase}
Основные возможности, которые предоставляет Firebase связанные с хранением, изменением и получением информации из встроенной базы данных. Firebase не имеет возможности для создания API, поэтому для решения вспомогательных задач, не связанных с базой данных необходим дополнительный API-cервер, который будет заниматься обработкой запросов и обращаться к Firestore по мере необходимости. Это позволит избежать проблем, связанных с невозможность выполнения некоторых операций на компьютере конечного пользователя. Так как самые ресурсозатратные операции выполняет Firebase, данный сервер может выполнять небольшой список операций, что очень полезно для данного проекта.
 
\subsection{Выбор языка программирования}
Так как было решено решено разработать веб-приложение, разработку необходимо вести на языке, который имеет поддержку для выполнения внутри браузера. К таковым можно отнести: JavaScript, TypeScript, Kotlin. По своей сути все варианты языков это JavaScript, так как браузер работает именно с ним, но выбор того или иного языка может сказаться на процессе разработки и дальнейшей поддержке.
 
Kotlin является наиболее молодым из тройки и из-за данного обстоятельства может вызвать трудности у разработчиков не знакомых с проектом, так как на текущий момент основное применение kotlin находит в android  разработке. Помимо этого механизм работы kotlina для веб-разработки создаёт лишние трудности, ввиду необходимости трансляции кода на Kotlin в рабочий JavaScript код.
 
TypeScript - язык программирования от компании Microsoft, как альтернатива JavaScript, расширяющая его возможности. 
Основной особенностью TypeScript можно считать систему типизации, так как в отличии от обычного JavaScript, она статическая.
Однако реализация типизации в TypeScript не решает проблемы динамической типизации JavaScript полностью, так как всё равно возможны случаи, когда переменная может иметь тип отличный от объявленного.
Также стоит отметить, что многие возможности TypeScript в том или ином виде приходят и в JavaScript, что делает TypeScript не лучшим выбором в долгосрочной перспективе.
 
\subsection{JavaScript}
Таким образом был сделан выбор в сторону языка JavaScript. Основной целью JavaScript является придание статическим HTML страницам динамичности, например отправка форм без перезагрузки страницы, изменение разметки страницы и стилей элементов на лету и многое другое. Программы написанные на JavaScript называются скриптами, они представляют из себя простой текстовый файл. Для работы JavaScript скриптов не нужна дополнительная подготовка или компиляция для работы. За выполнение JavaScript кода отвечает отдельная программа, которую обычно называют движком или «виртуальной машиной JavaScript». За счёт это JavaScript скрипты можно выполнять не только в браузере, но и на сервере при наличии JavaScript движка, который займется выполнением кода. Движок берёт код из скрипта, преобразует его в машинный код и затем выполняет его.
 
Современные браузеры в своём составе имеют собственные встроенные JavaScript движки для выполнения скриптов, например V8 в Chrome или SpiderMonkey в FireFox\cite{js3}. Хоть в основном реализации различных JavaScript движком похожи, но в них всё равно присутствуют различия, которые способны сказаться на работе JS кода в разных браузерах. Также могут быть различия в разных версиях одного и того же движка. Подобные нюансы необходимо учитывать при разработке веб-приложений. Для решения подобных проблем в JavaScript присутствуем механизм полифилов, позволяющий использовать новые и недоступные функции, на разных или старых браузерах. У JavaScript имеется собственный открытый стандарт ECMAScript, который должен быть реализован движками, но данный процесс происходить не моментально.
 
JavaScript отличается от многих современных языков программирования.
Эти отличия многими воспринимаются неоднозначно, так как они способны предложить богатую функциональность, но также эти отличия могут стать причиной непредвиденных неполадок в работе приложения. К таким отличиям можно отнести систему наследования.
В отличии многих других языков, в которых для наследования используется концепция классов, в JavaScript наследование опирается на понятие прототипа. Прототип~--- это объект, который имеет набор свойств и методов. Это значит, что объекты одного типа могут иметь разный набор свойств. Даже встроенные типы, структуры данных и функции представляют из себя обычные объекты с определенным набором свойств и функций.
Для получения возможностей массива объекту достаточно указать объект, реализующий методы массива как прототип. Также при помощи данной особенность можно расширить возможности базовых типов, расширяя их прототипы. В стандарте ECMAScript 6 ввели синтаксис для классического описания классов, но данное нововведение является синтаксическим сахаром, так как никак не меняет механизмы наследования, используемые в JavaScript.

Другой особенностью JavaScript является контекст выполнения кода (значение переменной this). Данная переменная ссылается на текущий объект при выполнении кода (окно браузера или конкретный объект).То, какой объект является текущим, может вызвать трудности. Если в других языках при выполнении вложенных методов классов переменная, ссылающаяся на основной объект, остаётся прежней (self в Python), то в подобной ситуации в JavaScript произойдёт смена контекста выполнения на глобальный. В результате этого произойдёт потеря текущего объекта при выполнении функции, что может привести к непредвиденным ошибкам. На данный момент для решения данной проблемы в стандарте ECMAScript 6 ввели стрелочные функции. Их основной особенностью является то, что данные функции сохраняю контекст, в котором их вызывают.
 
Также среди возможностей JavaScript можно выделить:
\begin{itemize}
 \item объекты с возможностью интроспекции;
 \item функции как объекты первого класса;
 \item автоматическое приведение типов;
 \item автоматическая сборка мусора;
 \item анонимные функции.
\end{itemize}
 
\subsection{Построение пользовательского интерфейса}
Далее встаёт вопрос визуализации пользовательского интерфейса. Любой сайт представляет из себя набор HTML документов, CSS стилей и JS скриптов. 
HTML, CSS отвечают за построение, структуру и визуальное оформление содержимого сайта. 
Хоть для создания динамического веб-приложения можно использовать только HTML, CSS и JavaSript, это подход вызовет дополнительные трудности при разработке большого количества динамический элементов и их дальнейшей поддержке. 
Поэтому для создания пользовательского интерфейса было решено использовать специализированные для этого библиотеки и фреймворки.
Среди них можно выделить:
\begin{itemize}
 \item AngularJS
 \item VueJS
 \item ReactJS
\end{itemize}
 
От AngularJS было решено отказаться по некоторым причинам: слишком большие размеры фреймворка, вызванные тем, что Angular предлагает много функциональных возможностей, однако эти возможности излишни для данного дипломного проекта. 
Также, в сравнении с React и Vue, Angular обладает проблемами с производительностью, сложен в тестировании и отладке кода из-за своих архитектурных решений.

Vue и React разделяют многие общие идеи и функции.
В данном случае выбор библиотеки обоснован личными предпочтениями разработчика.
React обладает большей поддержкой из-зв своей популярности и предлагает больше возможностей чем Vue, оставаясь небольшим по размеру. 
Также использование React может облегчить дальнейшую разработку мобильной версии приложения, благодаря React Native.
Эти две библиотеки, хоть и предназначены для разных вещей (веб и мобильная разработка соответственно), обладают схожей архитектурой и принципами работы. Этот факт является крайне полезным при портировании существующего веб-приложения на мобильные платформы IOS и Android.
 
\subsection{ReactJS}
ReactJS - JavaScript библиотека для создания пользовательских интерфейсов, разработанная компанией Facebook. React обладает хорошей производительностью и архитектурой для создания динамичных\linebreak веб-приложений. Среди основных преимуществ стоит выделить: виртуальный DOM и JSX.
 
\subsubsection{Виртуальный DOM}~\par
Основной концепцией и архитектурной особенностью React является виртуальный DOM\cite{react2}. 
Для взаимодействия с HTML документом используется Document Object Model~--- платформо независимый интерфейс, позволяющий программам и скриптам получить доступ к содержимому HTML, XHTML, XML документов, модифицировать их содержимое и оформление. 
Элементы документа представляются в качестве узлов со связями "родитель-потомок". 
Проблема DOM заключается в его неспособности эффективно работать с современными динамическими пользовательскими интерфейсами, так как обработка большого количества элементов может занимать слишком много времени. 
Данную проблему можно решить, используя различные ухищрения, но глобально это не решает проблемы разработки современных динамических интерфейсов. 
Решить эту проблему призвана концепция виртуального DOM. 
Виртуальный DOM не является общепринятой технологией или стандартом, это подход, который заключается в том, что взаимодействия происходят с DOM не напрямую. 
Код оперирует не DOM, а его легковесной копией. 
После обновления копии происходит процесс согласования реального DOM и виртуального. 
Во время данного процесса происходит сравнение текущего дерева элементов и нового, затем для изменённых элементов происходит перестроение и изменение структуры поддерева DOM. 
React берет на себя процесс сравнения деревьев, стараясь обеспечить наилучшую производительность и скорость. 
Однако React не решает всех проблем автоматически. 
Разработчику необходимо контролировать поток данных и проектировать интерфейс так, чтобы изменяемые элементы не имели больших поддеревьев DOM. 
Это позволит избежать излишних перестроек реального DOM, что значительно увеличивает производительность приложения.
 
\subsubsection{JSX}~\par
Для описание пользовательского интерфейса React используют JSX~--- расширение языка JavaScript. 
Он напоминает стандартный язык шаблонов, но обладает всей силой JavaScript. 
React не разделяет разные компоненты приложения, такие как представления и логика. 
Вместо этого React использует абстрактную структуру "компонент", которая объединяет в себе логику пользовательского интерфейса и сам интерфейс. 
Компонент является представлением DOM элементов в его виртуальной копии.
При работе приложения JSX преобразуется в специальную функцию для создания DOM элемента, которая затем транслируется при помощи Babel в обычный JavaScript.

 
\subsection{Технологии для серверной части}
Ни одно веб-приложение невозможно без серверной части, которая отвечает за работу с базой данных, обработку данных и выполнение роли API, и данный проект не исключение. При выборе средств разработки основными критериями были: знакомство разработчика с данными инструментами, возможность быстрого создания минимально необходимой функциональности, без необходимости поддержки большой кодовой базы, легковесность, возможность расширения в дальнейшем при наличии необходимости. Наилучшим выбором оказался язык программирования Python и библиотека Flask.
 
\subsubsection{Python}~\par
Python - высокоуровневый, многоплатформенный язык программирования общего назначения, является одним из наиболее востребованных языков, используемых при создании приложений, системных средств, в различных научных целях и веб-разработке\cite{python}. Данный язык ориентирован на повышение скорости разработки и читаемость кода. Среди достоинств и особенностей языка можно выделить:
\begin{itemize}
 \item скорость разработки;
 \item динамическая типизация;
 \item множество сторонних модулей и библиотек;
 \item высокая масштабируемость;
 \item многочисленное сообщество;
 \item встраиваемость.
\end{itemize}
 
Python поддерживает различные парадигмы программирования, среди них: структурное, объектно-ориентированное, функциональное, императивное и аспектно-ориентированное программирование. Python является динамически типизированным языком программирования, это означает что тип переменной определяется в момент присваивания значения этой переменной. Данная особенность позволяет ускорить процесс разработки при работе с данными переменных типов и изменяющемся окружении\cite{python2}. Другим важным плюсом Python является собственный пакетный менеджер, при помощи которого можно добавить сторонние модули для расширения возможностей языка, например модуль numpy добавляет множество классов и функций для выполнения математических исследований. Одним из главных преимуществ Python для веб-разработки является наличие большого количества популярных и функциональных библиотек. В сравнении со многими другими языками, например Java или \csharp{}, Python сочетает в себе как продвинутые возможности для разработки веб-приложений, так и простоту их создания, не нуждаясь в громоздких и сложных фреймворках.
 
При выборе были рассмотрены и недостатки языка, важнейшим из которых был недостаток в скорости. Python является языком с полной динамической типизацией, автоматическим  управлением памятью. Если на первый взгляд это может казаться преимуществом, то при разработке программ с повышенным требованием к эффективности, Python может значительно  проигрывать по скорости своим статическим братьям (C/C++, Java, Go). Что касается динамических собратьев (PHP, Ruby, JavaScript), то здесь дела обстоят намного лучше, Python в  большинстве случаев выполняет код быстрее за счет предварительной компиляции в байт-код и значительной части стандартной библиотеки, написанной на С.
 
\subsubsection{Flask}~\par
В качестве библиотеки для реализации серверной части был выбран Flask. 
Flask позиционирует себя как микро-фреймворк с возможностью к расширению.
Это выражается в том, что сама библиотека содержит очень небольшой набор базовых возможностей, а при необходимости тех или иных дополнительных функций нужно использовать сторонние совместимые инструменты. 
В противовес данному подходу можно выделить другую крайне популярную Python библиотеку~--- Django.
Она предлагает разработчику практически полный набор всего необходимо для написания веб-приложения, при этом накладывая ряд архитектурных ограничений.
Однако для того, чтобы получить необходимую для данного проекта функциональность (API-cервер) в Django, необходимо использовать дополнительную библиотеку~--- Django REST.
Она добавляет полноценный возможности REST API-cервера.
В совокупности Django и Django REST требуют крайне много ресурсов на разработку и поддержание, особенно при требовании небольшой функциональности. 
Поэтому Flask оказался предпочтительнее из-за своего размера и расширяемости.

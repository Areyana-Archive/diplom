{
  \newgeometry{top=1.25cm,bottom=1.25cm,right=1cm,left=2cm,twoside}
  \thispagestyle{empty}
  \setlength{\parindent}{0em}

  \newcommand{\lineunderscore}{\uline{\hspace*{\fill}}}

  \begin{center}
    Министерство образования Республики Беларусь\\
    Учреждение образования\\
    БЕЛОРУССКИЙ ГОСУДАРСТВЕННЫЙ УНИВЕРСИТЕТ \\
    ИНФОРМАТИКИ И РАДИОЭЛЕКТРОНИКИ\\[1em]
  

  \begin{minipage}{\textwidth}
    \begin{flushleft}
      \begin{tabular}{ l }
        Факультет: ФКСиС. Кафедра: Информатики. \\
        Специальность: 40 04 01 "Информатика и технологии программирования".
      \end{tabular}
    \end{flushleft}
  \end{minipage}\\[1em]

  \begin{minipage}{\textwidth}
    \begin{flushright}
      \begin{tabular}{p{0.40\textwidth}}
        УТВЕРЖДАЮ \\
        Заведующая кафедрой информатики \\
        \underline{\hspace*{7em}} Н. А. Волорова
        <<\underline{\hspace*{4ex}}>> \underline{\hspace*{5em}} 2020 г.
      \end{tabular}
    \end{flushright}
  \end{minipage}\\[1em]

  \text{ЗАДАНИЕ} \\
  \text{по дипломному проекту студента}

  {\text{Демидовича Ильи Вадимовича}}

  \end{center}

  1. Тема проекта: \textquote{Android-приложение "Агрегатор игровых магазинов"}~--- утверждена приказом по университету от 13 апреля 2021 г.  \No{} 827-с

  \vspace{1em}

  2. Срок сдачи студентом законченного проекта: \underline{\hspace*{5em}}

  \vspace{1em}

  3. Исходные данные к проекту:\\
\uline{Тип операционной системы – Android 5.1 и выше; Язык программирования – Kotlin; Среда разработки – Android Studio 3.6; СУБД –SQLite\hfill}\\
\uline{Назначение разработки: упрощение процесса поиска и просмотра предложений среди игровых магазинов.\hfill}\\
\uline{\hfill}\\
\uline{\hfill}\\
\uline{\hfill}\\
%\lineunderscore\\
%\lineunderscore\\
%\lineunderscore\\
%\lineunderscore\\

  \vspace{1em}

  4. Содержание пояснительной записки (перечень подлежащих разработке вопросов):
\uline{Введение\hfill}\\
\uline{\hfill}\\
\uline{\hfill}\\
\uline{\hfill}\\
\uline{\hfill}\\
\uline{\hfill}\\
\uline{\hfill}\\
\uline{\hfill}\\
% \lineunderscore\\
% \lineunderscore\\
% \lineunderscore\\
% \lineunderscore\\
% \lineunderscore\\
% \lineunderscore\\
% \lineunderscore\\
% \lineunderscore\\


  \clearpage
  \thispagestyle{empty}

  5. Перечень графического материала (с точным указанием обязательных чертежей):
  \uline{\hfill}\\
  \uline{\hfill}\\
  \uline{\hfill}\\
  \uline{\hfill}\\
  \uline{\hfill}\\
  \uline{\hfill}\\
  %\lineunderscore\\
  %\lineunderscore\\
  %\lineunderscore\\
  %\lineunderscore\\
  %\lineunderscore\\
  %\lineunderscore\\


  \vspace{1em}

  6. Содержание задания по технико-экономическому обоснованию:\\
  \uline{1. Описание функций, назначения и потенциальных пользователей ПО\hfill}\\
  \uline{2. Расчет затрат на разработку ПО\hfill}\\
  \uline{3. Экономический эффект при разработке ПО\hfill}
  \vspace{1em}

  ЗАДАНИЕ ВЫДАЛА \hfill{}  Е.\,Е.~Марченкова  

  \vspace{1em}

  \vfill

  \begin{center}
    КАЛЕНДАРНЫЙ ПЛАН
  \end{center}

  \begin{tabular}{| >{\centering}m{0.04\textwidth} 
                  | >{\centering}m{0.40\textwidth} 
                  | >{\centering}m{0.08\textwidth}
                  | >{\centering}m{0.19\textwidth}  
                  | >{\centering\arraybackslash}m{0.16\textwidth}|}
    \hline \No{} п/п & Наименование этапов дипломного проекта & Объем этапа, \% & Срок выполнения этапов & Примечание \\
    \hline 1 & \raggedright Анализ предметной области, разработка технического задания & 15 & 24.03 - 30.03 & \\
    \hline 2 & \raggedright Разработка функциональных требований, проектирование архитектуры программы & 20 & 31.03 - 15.04 & \\
    \hline 3 & \raggedright Разработка схемы программы, алгоритмов, схемы данных & 15 & 16.04 - 20.04 & \\
    \hline 4 & \raggedright Разработка программного средства & 30 & 21.04 - 12.05 & \\
    \hline 5 & \raggedright Тестирование и отладка & 10 & 13.05 - 15.05 & \\
    \hline 6 & \raggedright Оформление пояснительной записки и графического материала & 20 & 18.05 - 1.06 & \\
    \hline
  \end{tabular}

  \vspace{2em}

  Дата выдачи задания: \underline{\hspace*{5em}} \\
  Руководитель \hfill{} М.\,В.~Стержанов

  \vspace{1em}

  ЗАДАНИЕ ПРИНЯЛ К ИСПОЛНЕНИЮ \hfill{} И.\,В.~Демидович

  \restoregeometry
}